\documentclass[]{article}
\usepackage{lmodern}
\usepackage{amssymb,amsmath}
\usepackage{ifxetex,ifluatex}
\usepackage{fixltx2e} % provides \textsubscript
\ifnum 0\ifxetex 1\fi\ifluatex 1\fi=0 % if pdftex
  \usepackage[T1]{fontenc}
  \usepackage[utf8]{inputenc}
\else % if luatex or xelatex
  \ifxetex
    \usepackage{mathspec}
  \else
    \usepackage{fontspec}
  \fi
  \defaultfontfeatures{Ligatures=TeX,Scale=MatchLowercase}
\fi
% use upquote if available, for straight quotes in verbatim environments
\IfFileExists{upquote.sty}{\usepackage{upquote}}{}
% use microtype if available
\IfFileExists{microtype.sty}{%
\usepackage{microtype}
\UseMicrotypeSet[protrusion]{basicmath} % disable protrusion for tt fonts
}{}
\usepackage[margin=1in]{geometry}
\usepackage{hyperref}
\hypersetup{unicode=true,
            pdftitle={minst},
            pdfauthor={Z.CAI},
            pdfborder={0 0 0},
            breaklinks=true}
\urlstyle{same}  % don't use monospace font for urls
\usepackage{color}
\usepackage{fancyvrb}
\newcommand{\VerbBar}{|}
\newcommand{\VERB}{\Verb[commandchars=\\\{\}]}
\DefineVerbatimEnvironment{Highlighting}{Verbatim}{commandchars=\\\{\}}
% Add ',fontsize=\small' for more characters per line
\usepackage{framed}
\definecolor{shadecolor}{RGB}{248,248,248}
\newenvironment{Shaded}{\begin{snugshade}}{\end{snugshade}}
\newcommand{\AlertTok}[1]{\textcolor[rgb]{0.94,0.16,0.16}{#1}}
\newcommand{\AnnotationTok}[1]{\textcolor[rgb]{0.56,0.35,0.01}{\textbf{\textit{#1}}}}
\newcommand{\AttributeTok}[1]{\textcolor[rgb]{0.77,0.63,0.00}{#1}}
\newcommand{\BaseNTok}[1]{\textcolor[rgb]{0.00,0.00,0.81}{#1}}
\newcommand{\BuiltInTok}[1]{#1}
\newcommand{\CharTok}[1]{\textcolor[rgb]{0.31,0.60,0.02}{#1}}
\newcommand{\CommentTok}[1]{\textcolor[rgb]{0.56,0.35,0.01}{\textit{#1}}}
\newcommand{\CommentVarTok}[1]{\textcolor[rgb]{0.56,0.35,0.01}{\textbf{\textit{#1}}}}
\newcommand{\ConstantTok}[1]{\textcolor[rgb]{0.00,0.00,0.00}{#1}}
\newcommand{\ControlFlowTok}[1]{\textcolor[rgb]{0.13,0.29,0.53}{\textbf{#1}}}
\newcommand{\DataTypeTok}[1]{\textcolor[rgb]{0.13,0.29,0.53}{#1}}
\newcommand{\DecValTok}[1]{\textcolor[rgb]{0.00,0.00,0.81}{#1}}
\newcommand{\DocumentationTok}[1]{\textcolor[rgb]{0.56,0.35,0.01}{\textbf{\textit{#1}}}}
\newcommand{\ErrorTok}[1]{\textcolor[rgb]{0.64,0.00,0.00}{\textbf{#1}}}
\newcommand{\ExtensionTok}[1]{#1}
\newcommand{\FloatTok}[1]{\textcolor[rgb]{0.00,0.00,0.81}{#1}}
\newcommand{\FunctionTok}[1]{\textcolor[rgb]{0.00,0.00,0.00}{#1}}
\newcommand{\ImportTok}[1]{#1}
\newcommand{\InformationTok}[1]{\textcolor[rgb]{0.56,0.35,0.01}{\textbf{\textit{#1}}}}
\newcommand{\KeywordTok}[1]{\textcolor[rgb]{0.13,0.29,0.53}{\textbf{#1}}}
\newcommand{\NormalTok}[1]{#1}
\newcommand{\OperatorTok}[1]{\textcolor[rgb]{0.81,0.36,0.00}{\textbf{#1}}}
\newcommand{\OtherTok}[1]{\textcolor[rgb]{0.56,0.35,0.01}{#1}}
\newcommand{\PreprocessorTok}[1]{\textcolor[rgb]{0.56,0.35,0.01}{\textit{#1}}}
\newcommand{\RegionMarkerTok}[1]{#1}
\newcommand{\SpecialCharTok}[1]{\textcolor[rgb]{0.00,0.00,0.00}{#1}}
\newcommand{\SpecialStringTok}[1]{\textcolor[rgb]{0.31,0.60,0.02}{#1}}
\newcommand{\StringTok}[1]{\textcolor[rgb]{0.31,0.60,0.02}{#1}}
\newcommand{\VariableTok}[1]{\textcolor[rgb]{0.00,0.00,0.00}{#1}}
\newcommand{\VerbatimStringTok}[1]{\textcolor[rgb]{0.31,0.60,0.02}{#1}}
\newcommand{\WarningTok}[1]{\textcolor[rgb]{0.56,0.35,0.01}{\textbf{\textit{#1}}}}
\usepackage{graphicx,grffile}
\makeatletter
\def\maxwidth{\ifdim\Gin@nat@width>\linewidth\linewidth\else\Gin@nat@width\fi}
\def\maxheight{\ifdim\Gin@nat@height>\textheight\textheight\else\Gin@nat@height\fi}
\makeatother
% Scale images if necessary, so that they will not overflow the page
% margins by default, and it is still possible to overwrite the defaults
% using explicit options in \includegraphics[width, height, ...]{}
\setkeys{Gin}{width=\maxwidth,height=\maxheight,keepaspectratio}
\IfFileExists{parskip.sty}{%
\usepackage{parskip}
}{% else
\setlength{\parindent}{0pt}
\setlength{\parskip}{6pt plus 2pt minus 1pt}
}
\setlength{\emergencystretch}{3em}  % prevent overfull lines
\providecommand{\tightlist}{%
  \setlength{\itemsep}{0pt}\setlength{\parskip}{0pt}}
\setcounter{secnumdepth}{0}
% Redefines (sub)paragraphs to behave more like sections
\ifx\paragraph\undefined\else
\let\oldparagraph\paragraph
\renewcommand{\paragraph}[1]{\oldparagraph{#1}\mbox{}}
\fi
\ifx\subparagraph\undefined\else
\let\oldsubparagraph\subparagraph
\renewcommand{\subparagraph}[1]{\oldsubparagraph{#1}\mbox{}}
\fi

%%% Use protect on footnotes to avoid problems with footnotes in titles
\let\rmarkdownfootnote\footnote%
\def\footnote{\protect\rmarkdownfootnote}

%%% Change title format to be more compact
\usepackage{titling}

% Create subtitle command for use in maketitle
\providecommand{\subtitle}[1]{
  \posttitle{
    \begin{center}\large#1\end{center}
    }
}

\setlength{\droptitle}{-2em}

  \title{minst}
    \pretitle{\vspace{\droptitle}\centering\huge}
  \posttitle{\par}
    \author{Z.CAI}
    \preauthor{\centering\large\emph}
  \postauthor{\par}
      \predate{\centering\large\emph}
  \postdate{\par}
    \date{2021年11月9日}


\begin{document}
\maketitle

\begin{Shaded}
\begin{Highlighting}[]
\FunctionTok{setwd}\NormalTok{(}\StringTok{"C:/Users/CAI/Desktop/Deep\_Learning\_with\_R/MINST/MINST\_example"}\NormalTok{)}
\end{Highlighting}
\end{Shaded}

\hypertarget{the-mnist-data-example}{%
\section{The MNIST Data Example}\label{the-mnist-data-example}}

\begin{Shaded}
\begin{Highlighting}[]
\FunctionTok{library}\NormalTok{(keras)}

\NormalTok{mnist }\OtherTok{\textless{}{-}} \FunctionTok{dataset\_mnist}\NormalTok{()}
\end{Highlighting}
\end{Shaded}

\begin{verbatim}
## Warning in normalizePath(path.expand(path), winslash, mustWork):
## path[1]="C:\Users\CAI\anaconda3\envs\rstudio_conda/python.exe": 系统找不到
## 指定的文件。
\end{verbatim}

\begin{verbatim}
## Warning in normalizePath(path.expand(path), winslash, mustWork):
## path[1]="C:\Users\CAI\anaconda3\envs\tensorflow_2.6/python.exe": 系统找不到
## 指定的文件。
\end{verbatim}

\begin{verbatim}
## Warning in normalizePath(path.expand(path), winslash, mustWork):
## path[1]="C:\Users\CAI\anaconda3\envs\rstudio_conda/python.exe": 系统找不到
## 指定的文件。
\end{verbatim}

\begin{verbatim}
## Warning in normalizePath(path.expand(path), winslash, mustWork):
## path[1]="C:\Users\CAI\anaconda3\envs\tensorflow_2.6/python.exe": 系统找不到
## 指定的文件。
\end{verbatim}

\begin{verbatim}
## Loaded Tensorflow version 2.6.1
\end{verbatim}

\begin{Shaded}
\begin{Highlighting}[]
\NormalTok{train\_images }\OtherTok{\textless{}{-}}\NormalTok{ mnist}\SpecialCharTok{$}\NormalTok{train}\SpecialCharTok{$}\NormalTok{x}
\NormalTok{train\_labels }\OtherTok{\textless{}{-}}\NormalTok{ mnist}\SpecialCharTok{$}\NormalTok{train}\SpecialCharTok{$}\NormalTok{y}
\NormalTok{test\_images }\OtherTok{\textless{}{-}}\NormalTok{ mnist}\SpecialCharTok{$}\NormalTok{test}\SpecialCharTok{$}\NormalTok{x}
\NormalTok{test\_labels }\OtherTok{\textless{}{-}}\NormalTok{ mnist}\SpecialCharTok{$}\NormalTok{test}\SpecialCharTok{$}\NormalTok{y}
\end{Highlighting}
\end{Shaded}

\hypertarget{glimpse-the-structure-of-array-using-str-function}{%
\section{glimpse the structure of array using str()
function}\label{glimpse-the-structure-of-array-using-str-function}}

\begin{Shaded}
\begin{Highlighting}[]
\FunctionTok{str}\NormalTok{(train\_images)}
\end{Highlighting}
\end{Shaded}

\begin{verbatim}
##  int [1:60000, 1:28, 1:28] 0 0 0 0 0 0 0 0 0 0 ...
\end{verbatim}

\begin{Shaded}
\begin{Highlighting}[]
\FunctionTok{str}\NormalTok{(train\_labels)}
\end{Highlighting}
\end{Shaded}

\begin{verbatim}
##  int [1:60000(1d)] 5 0 4 1 9 2 1 3 1 4 ...
\end{verbatim}

\begin{Shaded}
\begin{Highlighting}[]
\FunctionTok{str}\NormalTok{(test\_images)}
\end{Highlighting}
\end{Shaded}

\begin{verbatim}
##  int [1:10000, 1:28, 1:28] 0 0 0 0 0 0 0 0 0 0 ...
\end{verbatim}

\begin{Shaded}
\begin{Highlighting}[]
\FunctionTok{str}\NormalTok{(test\_labels)}
\end{Highlighting}
\end{Shaded}

\begin{verbatim}
##  int [1:10000(1d)] 7 2 1 0 4 1 4 9 5 9 ...
\end{verbatim}

\hypertarget{the-network-architecture}{%
\section{the network architecture}\label{the-network-architecture}}

\begin{Shaded}
\begin{Highlighting}[]
\NormalTok{network }\OtherTok{\textless{}{-}} \FunctionTok{keras\_model\_sequential}\NormalTok{() }\SpecialCharTok{\%\textgreater{}\%}
  \FunctionTok{layer\_dense}\NormalTok{(}\AttributeTok{units =} \DecValTok{512}\NormalTok{, }\AttributeTok{activation =} \StringTok{"relu"}\NormalTok{, }\AttributeTok{input\_shape =} \FunctionTok{c}\NormalTok{(}\DecValTok{28} \SpecialCharTok{*} \DecValTok{28}\NormalTok{)) }\SpecialCharTok{\%\textgreater{}\%}
  \FunctionTok{layer\_dense}\NormalTok{(}\AttributeTok{units =} \DecValTok{10}\NormalTok{, }\AttributeTok{activation =} \StringTok{"softmax"}\NormalTok{)}
\end{Highlighting}
\end{Shaded}

\hypertarget{the-compilation-step}{%
\section{the compilation step}\label{the-compilation-step}}

\begin{Shaded}
\begin{Highlighting}[]
\NormalTok{network }\SpecialCharTok{\%\textgreater{}\%} \FunctionTok{compile}\NormalTok{(}
  \AttributeTok{optimizer =} \StringTok{"rmsprop"}\NormalTok{,}
  \AttributeTok{loss =} \StringTok{"categorical\_crossentropy"}\NormalTok{,}
  \AttributeTok{metrics =} \FunctionTok{c}\NormalTok{(}\StringTok{"accuracy"}\NormalTok{)}
\NormalTok{)}
\end{Highlighting}
\end{Shaded}

\hypertarget{preparing-the-image-data}{%
\section{preparing the image data}\label{preparing-the-image-data}}

\begin{Shaded}
\begin{Highlighting}[]
\NormalTok{train\_images }\OtherTok{\textless{}{-}} \FunctionTok{array\_reshape}\NormalTok{(train\_images, }\FunctionTok{c}\NormalTok{(}\DecValTok{60000}\NormalTok{, }\DecValTok{28} \SpecialCharTok{*} \DecValTok{28}\NormalTok{))}
\NormalTok{train\_images }\OtherTok{\textless{}{-}}\NormalTok{ train\_images }\SpecialCharTok{/} \DecValTok{255}

\NormalTok{test\_images }\OtherTok{\textless{}{-}} \FunctionTok{array\_reshape}\NormalTok{(test\_images, }\FunctionTok{c}\NormalTok{(}\DecValTok{10000}\NormalTok{, }\DecValTok{28} \SpecialCharTok{*} \DecValTok{28}\NormalTok{))}
\NormalTok{test\_images }\OtherTok{\textless{}{-}}\NormalTok{ test\_images }\SpecialCharTok{/} \DecValTok{255}
\end{Highlighting}
\end{Shaded}

\hypertarget{preparing-the-labels}{%
\section{preparing the labels}\label{preparing-the-labels}}

\begin{Shaded}
\begin{Highlighting}[]
\NormalTok{train\_labels }\OtherTok{\textless{}{-}} \FunctionTok{to\_categorical}\NormalTok{(train\_labels)}
\NormalTok{test\_labels }\OtherTok{\textless{}{-}} \FunctionTok{to\_categorical}\NormalTok{(test\_labels)}
\end{Highlighting}
\end{Shaded}

\hypertarget{train-the-network-keras-networks-fit-method}{%
\section{train the network (keras network's fit
method)}\label{train-the-network-keras-networks-fit-method}}

\begin{Shaded}
\begin{Highlighting}[]
\NormalTok{network }\SpecialCharTok{\%\textgreater{}\%} \FunctionTok{fit}\NormalTok{(}
\NormalTok{  train\_images,}
\NormalTok{  train\_labels,}
  \AttributeTok{epochs =} \DecValTok{5}\NormalTok{, }
  \AttributeTok{batch\_size =} \DecValTok{128}
\NormalTok{)}
\end{Highlighting}
\end{Shaded}



\end{document}
